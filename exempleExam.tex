\documentclass[addpoints,fr,biblatex,name,gradetable]{isae-exam}

\usepackage[utf8]{inputenc}
\addbibresource{exam.bib}

\usepackage{listings}
\lstset{language=[LaTeX]TeX, extendedchars=true, basicstyle=\ttfamily,
  commentstyle=\itshape, showstringspaces=false} 

\usepackage{verbatim}

\module{LV102}
\modulename{Espagno Facilo}
\uf{UF Langues et Civilisation}
\title[Ex. de cultura generala]{Exameno de cultura generala}
\date{1er janvier 2013}

\conditions{Cet examen d'Espagnolo Facilo est composé de
  \numquestions{} exercices indépendants. Le barême indiqué pour
  chaque exercice l'est à titre indicatif et sera de tout façon
  modifié à la testa del cliento comme on dit chez nous.

  Les seuls documents autorisés pour cet examen sont:

  \begin{itemize}
  \item le dernier numéro de \textit{Sport}
  \item la biographie non autorisée de Josep Guardiola
  \end{itemize}

  Merci de ne pas oublier votre nom, votre prénom et votre groupe
  hein.
}

\begin{document}

\begin{coverpages}
  \basiccoverpage{}
\end{coverpages}

\section{Introduction}
\label{sec:introduction}

Ce document est un exemple d'utilisation de la classe
\texttt{isae-exam}. Cette classe est prévue pour écrire des examens et
est dérivée de la classe \texttt{exam}. Les options, commandes,
environnements et options de personnalisation définis dans la classe
\texttt{exam} sont donc applicables pour \texttt{isae-exam}
(cf.~\cite{hirschorn11:_exam}).

\section{Options, mise en page, commandes et environnements
  utilisables}
\label{sec:options-etc}

\subsection{Options de la classe}
\label{sec:options}

Les options disponibles pour la classe sont les suivantes:

\begin{itemize}
\item options de choix de langue (la langue par défaut est la langue
  anglaise)!
  \begin{itemize}
  \item \lstinline!fr!: langue française
  \item \lstinline!fr-en!: langues française et anglaise chargées,
    langue française comme langue principale
  \item \lstinline!en-fr!: langues française et anglaise chargées,
    langue anglaise comme langue principale
  \end{itemize}
\item \lstinline!biblatex!: choix de \texttt{biblatex} pour gérer la
  bibliographie
\item choix de logo (par défaut, le logo est celui de l'ISAE):
  \begin{itemize}
  \item \lstinline!ensica!: logo de la formation ENSICA
  \item \lstinline!supaero!: logo de la formation SUPAERO
  \end{itemize}
\item \lstinline!name!: affichage d'un emplacement pour écrire le nom,
  le prénom et le groupe de l'étudiant sur la page de couverture
\item \lstinline!gradetable!: affichage d'un récapitulatif des points
  de chaque exercice sur la page de couverture
\end{itemize}

\subsection{Page de couverture, entêtes et pieds de pages}
\label{sec:page-de-couverture}

La page de couverture est définie par l'environnement
\lstinline!coverpages! (cf. section~\ref{sec:code-source-doc}). Une
macro, \lstinline!\basiccoverpage{}!, est prédéfinie et fournit une
page de couverture identique à celle de ce document (à condition
d'utiliser les bonnes options de classe). Tout cela est bien
évidemment personnalisable en redéfinissant les commandes appropriées.

Quelques commandes permettent de personnaliser la page de couverture
par défaut et les entêtes et pieds de pages:

\begin{itemize}
\item \lstinline!\module! permet de donner le code du module concerné
\item \lstinline!\modulename! permet de donner le nom complet du module
\item \lstinline!\uf! permet de donner le nom de l'Unité de Formation
  concernée
\item \lstinline!\conditions! permet de donner les conditions de
  l'examen
\end{itemize}

La commande \lstinline!title! a été redéfinie pour accepter un
argument optionnel qui sert de titre court (par défaut dans les
entêtes de page).

\subsection{Divers}
\label{sec:divers}

\begin{itemize}
\item la commande \lstinline!measurepage! permet de calculer l'espace
  vertical restant sur la page. Elle peut être utilisée avec
  l'environnement \lstinline!solution! pour déterminer la hauteur
  devant être laissée à disposition de l'étudiant pour écrire la
  réponse.
\end{itemize}

\section{Exemples de questions}
\label{sec:exemples-questions}

La suite de ce document est un exemple d'utilisation de
l'environnement \lstinline!questions!. En particulier, l'utilisation
de l'option \lstinline!answers! dans la classe permet de passer en
mode «~solution~».

\begin{questions}
\question[1] quel est le meilleur buteur de l'histoire du \textit{Futbol
  Club Barcelona} (\textit{més que un club})?

\begin{solution}[0pt]

  C'est Lionel «~j'ai pris des hormones~» Messi. Dans le source
  \LaTeX, l'option \lstinline![0pt]! de l'environnement
  \lstinline!solution! permet de préciser que l'on ne laisse pas de
  place pour répondre sur le sujet.
\end{solution}

\begin{marksdetail}
  \rowemph{} entête pour faire éventuellement joli &                              &        \\
                                                   & la réponse juste!            & 0      \\
                                                   & \bonus belle calligraphie    & 0.125  \\
                                                   & \malus écrit comme un cochon & -0.125 \\
\end{marksdetail}

\question[1\half] quel était le docteur attitré du \textit{Futbol
  Club Barcelona} (\textit{més que un club})?

\begin{solution}[5cm]

  Il paraît que ce serait le docteur Eufemiano Fuentes\ldots Là par
  contre, on a laissé 5cm pour répondre. Mais par contre, la
  solution ne prend que la place qu'il faut.
\end{solution}

\begin{marksdetail}
  \rowemph{} entête pour faire éventuellement joli &                              &        \\
                                                   & la réponse juste!            & 0      \\
                                                   & \bonus belle calligraphie    & 0.125  \\
                                                   & \malus écrit comme un cochon & -0.125 \\
\end{marksdetail}

\question[\half] on peut également faire des questions à choix multiples:
quel est le meilleur cloube de futchebol du monde?

\begin{checkboxes}
  \CorrectChoice le \textit{Futbol Club Barcelona}
  \choice le Barça
  \choice le FC Barcelone
\end{checkboxes}
\end{questions}

\ifprintanswers
  Pour finir, cette phrase n'apparaîtra que dans la correction de l'examen.
\else
  Pour finir, cette phrase n'apparaîtra pas dans la correction de l'examen.
\fi

\section{Code source du présent document}
\label{sec:code-source-doc}

\verbatiminput{exempleExam.tex}

\section{Code source de la classe}
\label{sec:code-source-classe}

\verbatiminput{isae-exam.cls}

\printbibliography

\end{document}
