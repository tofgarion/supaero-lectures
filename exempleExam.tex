\documentclass[addpoints,fr,biblatex,name,gradetable]{isae-exam}

\usepackage[utf8]{inputenc}
\addbibresource{exam.bib}

\usepackage{listings}
\lstset{language=[LaTeX]TeX, extendedchars=true, basicstyle=\ttfamily,
  commentstyle=\itshape, showstringspaces=false} 

\module{LV102}
\modulename{Espagno Facilo}
\uf{UF Langues et Civilisation}
\title[Ex. de cultura generala]{Exameno de cultura generala}
\date{1er janvier 2013}

\conditions{Cet examen d'Espagnolo Facilo est composé de
  \numquestions{} exercices indépendants. Le barême indiqué pour
  chaque exercice l'est à titre indicatif et sera de tout façon
  modifié à la testa del cliento comme on dit chez nous.

  Les seuls documents autorisés pour cet examen sont:

  \begin{itemize}
  \item le dernier numéro de \textit{Sport}
  \item la biographie non autorisée de Josep Guardiola
  \end{itemize}

  Merci de ne pas oublier votre nom, votre prénom et votre groupe
  hein.
}

\begin{document}

\begin{coverpages}
  \basiccoverpage{}
\end{coverpages}

\section{Introduction}
\label{sec:introduction}

Ce document est un exemple d'utilisation de la classe
\texttt{isae-exam}. Cette classe est prévue pour écrire des examens et
est dérivée de la classe \texttt{exam}. Les options, commandes,
environnements et options de personnalisation définis dans la classe
\texttt{exam} sont donc applicables pour \texttt{isae-exam}
(cf.~\cite{hirschorn11:_exam}).

\section{Exemples}
\label{sec:exemples}

La suite de ce document est un exemple d'utilisation de
l'environnement \lstinline!questions!. En particulier, l'utilisation
de l'option \lstinline!answers! dans la classe permet de passer en
mode «~solution~».

\begin{questions}
\question[1] quel est le meilleur buteur de l'histoire du \textit{Futbol
  Club Barcelona} (\textit{més que un club})?

\begin{solution}[0pt]

  C'est Lionel «~j'ai pris des hormones~» Messi. Dans le source
  \LaTeX, l'option \lstinline![0pt]! de l'environnement
  \lstinline!solution! permet de préciser que l'on ne laisse pas de
  place pour répondre sur le sujet.
\end{solution}

\question[1\half] quel était le docteur attitré du \textit{Futbol
  Club Barcelona} (\textit{més que un club})?

\begin{solution}[5cm]

  Il paraît que ce serait le docteur Eufemiano Fuentes\ldots Là par
  contre, on a laissé 5cm pour répondre. Mais par contre, la
  solution ne prend que la place qu'il faut.
\end{solution}

\question[\half] on peut également faire des questions à choix multiples:
quel est le meilleur cloube de futchebol du monde?

\begin{checkboxes}
  \CorrectChoice le \textit{Futbol Club Barcelona}
  \choice le Barça
  \choice le FC Barcelone
\end{checkboxes}
\end{questions}

\ifprintanswers
  Pour finir, ceci n'apparaîtra que dans la correction de l'examen.
\else
  Pour finir, ceci n'apparaîtra pas dans la correction de l'examen.
\fi

\printbibliography

\end{document}
