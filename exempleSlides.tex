\RequirePackage{atbegshi}
\documentclass[fr,biblatex]{isae-slides}

\usepackage[utf8]{inputenc}
\usepackage{wasysym}
\addbibresource{beamer-bib.bib}

\usepackage{coffee4}

\usepackage{listings}
\lstset{language=[LaTeX]TeX, extendedchars=true, basicstyle=\ttfamily,
  commentstyle=\itshape, showstringspaces=false} 


\begin{document}

\title[Template \LaTeX{} ISAE]{Présentation template \LaTeX{} ISAE}
\subtitle{Juste pour voir ce que cela donne\ldots}
\author{Christophe Garion}
\institute{DMIA -- ISAE}
\module[IN001]{\LaTeX{} à l'ISAE}
\subject{Slides to present my beamer class for ISAE slides.}
\keywords{ISAE, LaTeX, beamer}
\date{}

\begin{frame}{}
  \titlepage
\end{frame}

\begin{frame}{Plan}
  \tableofcontents[part=1,pausesections]
\end{frame}

\AtBeginSection[]
{
  \begin{frame}<beamer>
    \frametitle{Plan}
    \tableofcontents[current]
  \end{frame}
}

\part<presentation>{Main Part}

\section{Généralités}
\label{sec:gen}

\subsection{Une sous-section juste pour vérifier...}
\label{sec:sub}

% Transparent ------------------------------------------------ slide -
\begin{frame}{Un exemple de transparent simple}

\vfill

Voici un transparent classique:

\begin{itemize}
\item le bandeau supérieur contient le titre du transparent. Admirez
  l'effort pour intégrer le dégradé sans utiliser d'image \smiley
\item il n'y a pas de logo ISAE dans le bandeau supérieur pour gagner
  en lisibilité
\end{itemize}

\begin{enumerate}
\item les items sont représentés par des boules en dégradé
\item idem pour les items numérotés
\end{enumerate}

\vfill

\end{frame}
% ------------------------------------------------------------ slide -

\section{Options de la classe}
\label{sec:class}

% Transparent ------------------------------------------------ slide -
\begin{frame}
\frametitle{Options de mise en forme}

\vfill

Les options classiques de Beamer peuvent être utilisées
(\texttt{smaller} etc.). Les options de mise en forme des transparents
sont les suivantes:

\begin{itemize}
\item si rien n'est précisé, les transparents
  utilisent la couleur, les dégradés et les \textit{overlays}
\item \lstinline!trans!: les transparents utilisent la couleur, les
  dégradés, mais pas les \textit{overlays} (pour un rétroprojecteur
  par exemple)
\item \lstinline!handout!: les transparents sont en N\&B, l'image de
  la page de garde est simplifiée et les bandeaux de titre sont en
  noir sur fond blanc. Les transparents sont produits en 4 par page
\item \lstinline!notes!: idem à \lstinline!handout!, mais une page de
  note pour que les étudiants puissent prendre des notes personnelles
  est automatiquement ajoutée à chaque transparent
\end{itemize}

Le \textit{Makefile} construit tous les exemples correspondants à
partir de ce fichier.

\vfill

\end{frame}
% ------------------------------------------------------------ slide -

% Transparent ------------------------------------------------ slide -
\begin{frame}
\frametitle{Option de langues}

\vfill

Le paquetage \lstinline!babel! est utilisé. Par défaut, l'anglais est
la langue sélectionnée. Pour utiliser d'autres langues, on utilise les
options suivantes:

\begin{itemize}
\item \lstinline!fr!: langue française
\item \lstinline!fr-en!: langues française et anglaise chargées,
  langue française comme langue principale
\item \lstinline!en-fr!: langues française et anglaise chargées,
  langue anglaise comme langue principale
\end{itemize}

\vfill

Dans le cas d'une utilisation multilangues, la commande
\lstinline!\\setlanguage! permet de changer de langue dans le document
(utiliser alors \lstinline!french! et \lstinline!english! comme noms
de langues).

\vfill

\end{frame}
% ------------------------------------------------------------ slide -

\section{Blocs}
\label{sec:blocs}

% Transparent ------------------------------------------------ slide -
\begin{frame}
\frametitle{Différents blocs}

\vfill

\begin{block}{Mon bloc}

Un exemple de bloc simple. Normalement, pas (plus?) de problèmes avec
\texttt{itemize}:

\begin{itemize}
\item premier item
\item second item
\end{itemize}  
\end{block}

\vfill

\begin{definition}[Ma définition]
Les définitions, théorèmes etc. fonctionnent. Il faut positionner la
langue via les options de la classe pour avoir les noms en français.
\end{definition}

\vfill

\begin{example}
Les blocs \texttt{example} et \texttt{alerted} sont définis comme des
blocs «~normaux~».
\end{example}

\vfill

\end{frame}
% ------------------------------------------------------------ slide -

\section{Bibliographie}
\label{sec:biblio}

% Transparent ------------------------------------------------ slide -
\begin{frame}
\frametitle{Bibliographie sur un seul transparent}

J'utilise biblatex pour la bibliographie. Il reste à changer le style
beamer pour BibTeX car pour l'instant, il n'est pas très joli.

Une commande permet de n'insérer que des références particulières:

\vspace{1cm}

\biblioinline{bouveret12:_beamer}

\end{frame}
% ------------------------------------------------------------ slide -

% Transparent ------------------------------------------------ slide -
\begin{frame}
\frametitle{Bibliographie}

La commande \lstinline!biblio! permet d'insérer toutes les références
qui ont été utilisées jusqu'à présent:



\nocite{mittelbach04:_latex}
\biblio{}

\end{frame}
% ------------------------------------------------------------ slide -

\section{Ce qu'il reste à faire\ldots}
\label{sec:todo}

% Transparent ------------------------------------------------ slide -
\begin{frame}
\frametitle{TODO!}

\begin{itemize}
\item vérifier que les transparents «~\textit{continued}~»
  fonctionnent bien
\item vérifier que les transparents longs type bibliographie sont bien
  séparés sur plusieurs pages
\item travailler sur l'aspect de la bibliographie (fichier bbx de
  \texttt{biblatex} pour Beamer, fontes utilisées, petit item
  apparaissant à chaque fois etc.)
\item entendre les complaintes de tous les collègues parce que dans le
  bandeau inférieur, ils auraient voulu avoir le nom de leur chien
  0.75 cm à côté du nom du cours \smiley

\onslide<all:2>{
\item juste pour le fun, une petite tache de café \smiley

\cofeAm{0.5}{0.5}{0}{0.5cm}{0.5cm}

}
\end{itemize}

\end{frame}
% ------------------------------------------------------------ slide -

\end{document}
