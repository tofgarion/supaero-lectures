\RequirePackage{atbegshi}
\documentclass{beamer}

\usepackage[utf8]{inputenc}
\usepackage{ae,aecompl,aeguill}
\usepackage{etex}
\usepackage{wasysym}
\usepackage{hyperref}
\usepackage[french]{babel}
\uselanguage{French}
\languagepath{French}


\usetheme{isae}

\begin{document}

\title[Template \LaTeX{} ISAE]{Présentation template \LaTeX{} ISAE}
\subtitle{Juste pour voir ce que cela donne\ldots}
\author{Christophe Garion}
\institute{DMIA -- ISAE}
\module[IN001]{\LaTeX{} à l'ISAE}
\subject{Slides to present my beamer class for ISAE slides.}
\keywords{ISAE, LaTeX, beamer}
\date{}

\begin{frame}{}
  \titlepage
\end{frame}

\begin{frame}{Plan}
  \tableofcontents[part=1,pausesections]
\end{frame}

\AtBeginSection[]
{
  \begin{frame}<beamer>
    \frametitle{Plan}
    \tableofcontents[current]
  \end{frame}
}

\part<presentation>{Main Part}

\section{Généralités}
\label{sec:gen}

\subsection{Une sous-section juste pour vérifier...}
\label{sec:sub}

% Transparent ------------------------------------------------ slide -
\begin{frame}{Un exemple de transparent simple}

\vfill

Voici un transparent classique:

\begin{itemize}
\item le bandeau supérieur contient le titre du transparent. Admirez
  l'effort pour intégrer le dégradé sans utiliser d'image \smiley
\item il n'y a pas de logo ISAE dans le bandeau supérieur pour gagner
  en lisibilité
\end{itemize}

\begin{enumerate}
\item les items sont représentés par des boules en dégradé
\item idem pour les items numérotés
\end{enumerate}

\vfill

\end{frame}
% ------------------------------------------------------------ slide -

\section{Blocs}
\label{sec:blocs}

% Transparent ------------------------------------------------ slide -
\begin{frame}
\frametitle{Différents blocs}

\vfill

\begin{block}{Mon bloc}

Un exemple de bloc simple. Normalement, pas (plus?) de problèmes avec
\texttt{itemize}:

\begin{itemize}
\item premier item
\item second item
\end{itemize}  
\end{block}

\vfill

\begin{definition}[Ma définition]
Les définitions, théorèmes etc. fonctionnent. Il faut positionner la
langue (cf. préambule) pour avoir les noms en français. Ce sera fait
automatiquement dans la classe finale.
\end{definition}

\vfill

\begin{example}
Les blocs \texttt{example} et \texttt{alerted} sont définis comme des
blocs «~normaux~».
\end{example}

\vfill


\end{frame}
% ------------------------------------------------------------ slide -

\end{document}
